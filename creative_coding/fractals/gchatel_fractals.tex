\documentclass[9pt]{beamer}

\usepackage[utf8x]{inputenc}
\usepackage[english]{babel}
\usepackage{amsmath, amsfonts, amssymb}
\usepackage{color}
\usepackage{xcolor}
\usepackage{tikz}
\usetikzlibrary{positioning,shapes,shadows,arrows,snakes}
\usepackage{listliketab}
\usepackage{shuffle}
\usepackage{xargs}
\usepackage{multirow}
\usepackage{pgfplots}
\usepackage{csquotes}
\usepackage{verbatim}

\definecolor{BlueGreen}{cmyk}{0.85,0,0.33,0}
\definecolor{RawSienna}{cmyk}{0,0.72,1,0.45}
\definecolor{gold}{rgb}{1.,0.84,0.}
\definecolor{dgreen}{rgb}{0.,0.6,0.}

\definecolor{Noir}{RGB}{0,0,0}
\definecolor{Rouge}{RGB}{205,35,38}
\definecolor{Bleu}{RGB}{2,60,195}
\definecolor{Bleu1}{RGB}{121,176,197}
\definecolor{Vert}{RGB}{23,103,1}
\definecolor{VertOlive}{RGB}{112,141,35}
\definecolor{Orange}{RGB}{255,113,15}
\definecolor{RoseBonbon}{RGB}{249,66,158}
\definecolor{Marron}{RGB}{193,88,50}

\definecolor{mygreen}{RGB}{23,103,1}

\newcommand{\red}[1]{\textcolor{red}{#1}}
\newcommand{\blue}[1]{\textcolor{blue}{#1}}
\newcommand{\green}[1]{\textcolor{mygreen}{#1}}
\newcommand{\bluealert}[2]{\textcolor<#1>{blue}{#2}}

\tikzstyle{alert} = [color=red, line width = 1.5]
\tikzstyle{bluealert} = [color=blue, line width =1.5]
\tikzstyle{big} = [line width = 1.5]
\tikzstyle{Point} = [fill, radius=0.08]
\tikzstyle{RedPoint} = [fill, radius=0.09, color = red]


\tikzstyle{Red} = [color = red]
\tikzstyle{Blue} = [color = blue]
\tikzstyle{Green} = [color = Vert]
\tikzstyle{Gray} = [color = gray]

\definecolor{violet}{rgb}{.5,.1,.9}


\usetheme{Boadilla}
\title{Fractal animation introduction}
\author[G. Châtel]{Grégory Châtel\\\vspace{0.3cm}@rodgzilla\\github.com/rodgzilla}
\date{October 18th, 2018}

\setbeamertemplate{footline}[frame number]{}
\setbeamertemplate{navigation symbols}{}

\begin{document}

%%%%%%%%%%%%%%%%%%%%%%%%%%%%%%%%%%%%%%%%%%%%%%%%%%%%%%%%%%%%%%%%%%%%%%
\begin{frame}

  \maketitle

\end{frame}
%%%%%%%%%%%%%%%%%%%%%%%%%%%%%%%%%%%%%%%%%%%%%%%%%%%%%%%%%%%%%%%%%%%%%%

\section{Introduction}

%%%%%%%%%%%%%%%%%%%%%%%%%%%%%%%%%%%%%%%%%%%%%%%%%%%%%%%%%%%%%%%%%%%%%%
\begin{frame}

  \frametitle{What are fractals?}

  \begin{center}
    \includegraphics[width = 8cm]{images/koch_snowflake.png}
  \end{center}

\end{frame}
%%%%%%%%%%%%%%%%%%%%%%%%%%%%%%%%%%%%%%%%%%%%%%%%%%%%%%%%%%%%%%%%%%%%%%

\section{Fractal examples}

%%%%%%%%%%%%%%%%%%%%%%%%%%%%%%%%%%%%%%%%%%%%%%%%%%%%%%%%%%%%%%%%%%%%%%
\begin{frame}
  \frametitle{Koch snowflake}

  \begin{center}
    \includegraphics[width = 8cm]{images/koch_building.png}
  \end{center}


\end{frame}
%%%%%%%%%%%%%%%%%%%%%%%%%%%%%%%%%%%%%%%%%%%%%%%%%%%%%%%%%%%%%%%%%%%%%%

%%%%%%%%%%%%%%%%%%%%%%%%%%%%%%%%%%%%%%%%%%%%%%%%%%%%%%%%%%%%%%%%%%%%%%
%% \begin{frame}
%%   \frametitle{Pythagora tree}

%% \end{frame}
%%%%%%%%%%%%%%%%%%%%%%%%%%%%%%%%%%%%%%%%%%%%%%%%%%%%%%%%%%%%%%%%%%%%%%

%%%%%%%%%%%%%%%%%%%%%%%%%%%%%%%%%%%%%%%%%%%%%%%%%%%%%%%%%%%%%%%%%%%%%%
\begin{frame}
  \begin{center}
    {\Huge \bf{Pythagora's tree demo}}
  \end{center}
\end{frame}
%%%%%%%%%%%%%%%%%%%%%%%%%%%%%%%%%%%%%%%%%%%%%%%%%%%%%%%%%%%%%%%%%%%%%%

%%%%%%%%%%%%%%%%%%%%%%%%%%%%%%%%%%%%%%%%%%%%%%%%%%%%%%%%%%%%%%%%%%%%%%
\begin{frame}
  \frametitle{Mandelbrot set}

  \begin{center}
    \includegraphics[width = 4cm]{images/mandelbrot_simple.png}
  \end{center}
\end{frame}
%%%%%%%%%%%%%%%%%%%%%%%%%%%%%%%%%%%%%%%%%%%%%%%%%%%%%%%%%%%%%%%%%%%%%%

%%%%%%%%%%%%%%%%%%%%%%%%%%%%%%%%%%%%%%%%%%%%%%%%%%%%%%%%%%%%%%%%%%%%%%
\begin{frame}[fragile]
  \frametitle{Complex numbers and complex plane}

  \begin{align*}
    c &= ai + b\\
    i^{2} &= -1
  \end{align*}

  The \emph{norm} of a complex number is its euclidiean distance to 0:

  \begin{align*}
    c &= ai + b \\
    |c| &= \sqrt{a^{2}+b^{2}} \\
  \end{align*}

  \begin{verbatim}
In [1]: c = complex(1, 2)

In [2]: c
Out[2]: (1+2j)

In [3]: c + complex(0,3)
Out[3]: (1+5j)

In [4]: c * c
Out[4]: (-3+4j)
  \end{verbatim}

\end{frame}
%%%%%%%%%%%%%%%%%%%%%%%%%%%%%%%%%%%%%%%%%%%%%%%%%%%%%%%%%%%%%%%%%%%%%%

%%%%%%%%%%%%%%%%%%%%%%%%%%%%%%%%%%%%%%%%%%%%%%%%%%%%%%%%%%%%%%%%%%%%%%
\begin{frame}
  \frametitle{Mandelbrot set}

  \framesubtitle{Function to iterate}

  The formula that generates everything is the following one:

  \begin{align*}
    z_{0} &= c \\
    z_{n + 1} &= z_{n}^{2} + c
  \end{align*}

  For each pixel $(x, y)$ of the screen, we compute the corresponding
  complex number $c_{x,y}$.

  \bigskip

  Now we compute a fixed number of terms of the sequence above
  starting with $z_{0} = c_{x,y}$.

  \begin{align*}
    z_{0} &= c_{x, y} \\
    z_{1} &= c_{x, y}^{2} + c_{x, y} \\
    z_{2} &= z_{1}^{2} + c_{x, y} = (c_{x, y}^{2} + c_{x, y})^{2} + c_{x, y} \\
    \dots \\
    z_{100} &= z_{99}^{2} + c_{x, y}
  \end{align*}

\end{frame}
%%%%%%%%%%%%%%%%%%%%%%%%%%%%%%%%%%%%%%%%%%%%%%%%%%%%%%%%%%%%%%%%%%%%%%

%%%%%%%%%%%%%%%%%%%%%%%%%%%%%%%%%%%%%%%%%%%%%%%%%%%%%%%%%%%%%%%%%%%%%%
\begin{frame}[fragile]
  \frametitle{Function interpolation with Python}

  \begin{center}
    \begin{lstlisting}
f_square = lambda x: x ** 2
f_cube   = lambda x: x ** 3
    \end{lstlisting}
  \end{center}

  \begin{center}
    \includegraphics[width = 6cm]{images/two_functions.png}
  \end{center}

\end{frame}
%%%%%%%%%%%%%%%%%%%%%%%%%%%%%%%%%%%%%%%%%%%%%%%%%%%%%%%%%%%%%%%%%%%%%%

%%%%%%%%%%%%%%%%%%%%%%%%%%%%%%%%%%%%%%%%%%%%%%%%%%%%%%%%%%%%%%%%%%%%%%
\begin{frame}
  \frametitle{Mandelbrot set}

  mandelbrot image.

\end{frame}
%%%%%%%%%%%%%%%%%%%%%%%%%%%%%%%%%%%%%%%%%%%%%%%%%%%%%%%%%%%%%%%%%%%%%%

\end{document}
