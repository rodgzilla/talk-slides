\documentclass[9pt]{beamer}

\usepackage[utf8x]{inputenc}
\usepackage[english]{babel}
\usepackage{amsmath, amsfonts, amssymb}
\usepackage{color}
\usepackage{xcolor}
\usepackage{tikz}
\usetikzlibrary{positioning,shapes,shadows,arrows,snakes}
\usepackage{listliketab}
\usepackage{shuffle}
\usepackage{xargs}
\usepackage{multirow}
\usepackage{pgfplots}
\usepackage{csquotes}
\usepackage{verbatim}

\definecolor{BlueGreen}{cmyk}{0.85,0,0.33,0}
\definecolor{RawSienna}{cmyk}{0,0.72,1,0.45}
\definecolor{gold}{rgb}{1.,0.84,0.}
\definecolor{dgreen}{rgb}{0.,0.6,0.}

\definecolor{Noir}{RGB}{0,0,0}
\definecolor{Rouge}{RGB}{205,35,38}
\definecolor{Bleu}{RGB}{2,60,195}
\definecolor{Bleu1}{RGB}{121,176,197}
\definecolor{Vert}{RGB}{23,103,1}
\definecolor{VertOlive}{RGB}{112,141,35}
\definecolor{Orange}{RGB}{255,113,15}
\definecolor{RoseBonbon}{RGB}{249,66,158}
\definecolor{Marron}{RGB}{193,88,50}

\definecolor{mygreen}{RGB}{23,103,1}

\newcommand{\red}[1]{\textcolor{red}{#1}}
\newcommand{\blue}[1]{\textcolor{blue}{#1}}
\newcommand{\green}[1]{\textcolor{mygreen}{#1}}
\newcommand{\bluealert}[2]{\textcolor<#1>{blue}{#2}}

\tikzstyle{alert} = [color=red, line width = 1.5]
\tikzstyle{bluealert} = [color=blue, line width =1.5]
\tikzstyle{big} = [line width = 1.5]
\tikzstyle{Point} = [fill, radius=0.08]
\tikzstyle{RedPoint} = [fill, radius=0.09, color = red]


\tikzstyle{Red} = [color = red]
\tikzstyle{Blue} = [color = blue]
\tikzstyle{Green} = [color = Vert]
\tikzstyle{Gray} = [color = gray]

\definecolor{violet}{rgb}{.5,.1,.9}


\usetheme{Boadilla}
\title{Importance of dataset for learning algorithms}
\author[G. Châtel]{Grégory Châtel\\\vspace{0.3cm}Disaitek\\Intel AI Software Innovator\\\vspace{0.3cm}@rodgzilla\\github.com/rodgzilla}
\date{September 25th, 2018}

\setbeamertemplate{footline}[frame number]{}
\setbeamertemplate{navigation symbols}{}

\begin{document}

%%%%%%%%%%%%%%%%%%%%%%%%%%%%%%%%%%%%%%%%%%%%%%%%%%%%%%%%%%%%%%%%%%%%%%
\begin{frame}

  \maketitle

\end{frame}
%%%%%%%%%%%%%%%%%%%%%%%%%%%%%%%%%%%%%%%%%%%%%%%%%%%%%%%%%%%%%%%%%%%%%%

%%%%%%%%%%%%%%%%%%%%%%%%%%%%%%%%%%%%%%%%%%%%%%%%%%%%%%%%%%%%%%%%%%%%%%
\begin{frame}

  \tableofcontents

\end{frame}
%%%%%%%%%%%%%%%%%%%%%%%%%%%%%%%%%%%%%%%%%%%%%%%%%%%%%%%%%%%%%%%%%%%%%%

\section{Machine learning basics}

%%%%%%%%%%%%%%%%%%%%%%%%%%%%%%%%%%%%%%%%%%%%%%%%%%%%%%%%%%%%%%%%%%%%%%
\begin{frame}

  \frametitle{Machine learning}

  \framesubtitle{Supervised learning}

  Machine learning is a subfield of artificial intelligence.

  \bigskip

  \begin{description}
    \item[Intuitively] We want to \emph{learn from} and \emph{make predictions
    on} data.

    \medskip

    \item[Technically] We want to build a model that approximate well
      (\textit{e.g.} minimize a loss function) an unknown function for
      which we only have limited observations.
  \end{description}

  \bigskip

  To do this, we usually need a lot of \emph{data}.

\end{frame}
%%%%%%%%%%%%%%%%%%%%%%%%%%%%%%%%%%%%%%%%%%%%%%%%%%%%%%%%%%%%%%%%%%%%%%o


\section{Popular ML tasks and their dataset}

%%%%%%%%%%%%%%%%%%%%%%%%%%%%%%%%%%%%%%%%%%%%%%%%%%%%%%%%%%%%%%%%%%%%%%
\begin{frame}

  \frametitle{Popular datasets for computer vision}

  \begin{description}[labelwidth=\widthof{bf series 2017, JFT-300M}]
    \setlength{\itemsep}{8pt}
    \item[1990, Statlog] $\sim$2k outdoor images
    \item[1998, MNIST] 60k B&W images of handwritten digits
    \item[2005, LabelMe] $\sim$187k scenes images
    \item[2009, ImageNet] $\sim$14M color images
    \item[2017, JFT-300M] $\sim$300M color images (internal dataset @ Google)
  \end{description}

\end{frame}
%%%%%%%%%%%%%%%%%%%%%%%%%%%%%%%%%%%%%%%%%%%%%%%%%%%%%%%%%%%%%%%%%%%%%%

%%%%%%%%%%%%%%%%%%%%%%%%%%%%%%%%%%%%%%%%%%%%%%%%%%%%%%%%%%%%%%%%%%%%%%
\begin{frame}

  \frametitle{Popular datasets for computer vision}

  \begin{description}[labelwidth=\widthof{bf series 2017, JFT-300M}]
    \setlength{\itemsep}{8pt}
    \item[1990, Statlog] $\sim$2k outdoor images
    \item[1998, MNIST] 60k B&W images of handwritten digits
    \item[2005, LabelMe] $\sim$187k scenes images
    \item[2009, ImageNet] $\sim$14M color images
    \item[2017, JFT-300M] $\sim$300M color images (internal dataset @ Google)
  \end{description}

\end{frame}
%%%%%%%%%%%%%%%%%%%%%%%%%%%%%%%%%%%%%%%%%%%%%%%%%%%%%%%%%%%%%%%%%%%%%%

%%%%%%%%%%%%%%%%%%%%%%%%%%%%%%%%%%%%%%%%%%%%%%%%%%%%%%%%%%%%%%%%%%%%%%
\begin{frame}

\end{frame}
%%%%%%%%%%%%%%%%%%%%%%%%%%%%%%%%%%%%%%%%%%%%%%%%%%%%%%%%%%%%%%%%%%%%%%

\section{Specific problem of dataset for NLP tasks}

%%%%%%%%%%%%%%%%%%%%%%%%%%%%%%%%%%%%%%%%%%%%%%%%%%%%%%%%%%%%%%%%%%%%%%
\begin{frame}

\end{frame}
%%%%%%%%%%%%%%%%%%%%%%%%%%%%%%%%%%%%%%%%%%%%%%%%%%%%%%%%%%%%%%%%%%%%%%

\section{Data efficiency}

\subsection{Model pre-training}

%%%%%%%%%%%%%%%%%%%%%%%%%%%%%%%%%%%%%%%%%%%%%%%%%%%%%%%%%%%%%%%%%%%%%%
\begin{frame}

\end{frame}
%%%%%%%%%%%%%%%%%%%%%%%%%%%%%%%%%%%%%%%%%%%%%%%%%%%%%%%%%%%%%%%%%%%%%%

\subsection{Semi supervision}

%%%%%%%%%%%%%%%%%%%%%%%%%%%%%%%%%%%%%%%%%%%%%%%%%%%%%%%%%%%%%%%%%%%%%%
\begin{frame}

\end{frame}
%%%%%%%%%%%%%%%%%%%%%%%%%%%%%%%%%%%%%%%%%%%%%%%%%%%%%%%%%%%%%%%%%%%%%%

%% Rappel des fondamentaux (optimisation du fonction de perte qui nécessite de confronter des données labelées avec les prédictions des modèles).
%% Présentation de la taille et de la qualité des jeux de données pour les disciplines classiques du machine learning.
%% Problématique en NLP pour la qualité et la quantité de données labelées
%% Solution : Utilisation des données non labelées pour améliorer les modèles
%% Pré-entrainement des modèles sur des tâches non-supervisées
%%  Semi-supervision en utilisant les données non labelées

\end{document}
